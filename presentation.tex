\documentclass{beamer}

\usepackage{pdfpages} % for the code of conduct

\title{\input{title.txt}}
\author{Markus A.G. Amano}
\subtitle{arXiv:2308.11686 \\{\tiny accepted for publication with Progress in Particle and Nuclear Physics}}
\institute{Yamagata University (as a JSPS Fellow)}

\begin{document}

{
\setbeamercolor{background canvas}{bg=}
\includepdf[pages=1]{codeofconduct.pdf}
}

\maketitle

\section{Two Motivating Questions}

\begin{itemize}
  \item What is the hydrodynamic description of QCD in the Quark Gluon Plasma phase?
  \item How does vorticity affect Quark Gluon Plasma matter?
\end{itemize}

% Keep the audience in mind both when preparing your talk and when presenting.
% The Rising Researchers Seminars are aimed at a broad nuclear theory audience.
% So you should assume that your audience has a good strong physics foundation,
% but you cannot assume that they know any particular nuclear physics detail
% beyond undergraduate physics, since they will all have pursued different
% graduate studies, learning different specific techniques and engaging with
% different tangential physics fields. In addition, remember that this series
% often has experimentalists in the audience.

% Provide a broad introduction that sets up the problem or question your research
% is addressing. Aim to provide a broad enough introduction that every audience
% member will feel they have understood in order to lay a strong foundation for
% the specifics of your research. Remember, it’s important to include the
% motivation for your research in the broader nuclear context, i.e., tell the
% audience why they should spend the next 40 min. listening to your talk instead
% of catching up on sleep.

% Don’t fall into the trap of thinking something is “too basic” to explain. We
% often feel that we are explaining something too basic and that the established
% senior researchers in the audience will be bored. This is a false impression,
% you can ask any of them. The less mental energy expended trying to recall
% “basic” concepts from the last time they taught it 20 years ago, the more
% energy they’ll have for understanding your actual research.

% Avoid jargon and acronyms. This is one of the biggest pitfalls for junior and
% senior researchers alike. If you do need to use them in your talk, always
% define them before you use them, even if they seem common. It’s generally good
% practice to remind your audience of the definition throughout the talk,
% particularly if it was defined early in your talk.

% Consider your color schemes carefully. One in twelve men are color blind (1 in
% 200 women). This is predominantly a red/green deficiency, so try not to use red
% and green in the same plot or text block. Here is a good resource to understand
% this better and get some ideas for workarounds. Also try to avoid any of the
% standard green, cyan, and magenta colors on a white background as they do not
% project well, even for people who are not color blind.

\begin{frame}
  Intro here
\end{frame}

\section{My Research}

\begin{frame}
  My Research here
\end{frame}

\section{Conclusions}

\begin{frame}
  Conclusions here 
\end{frame}

\end{document}
